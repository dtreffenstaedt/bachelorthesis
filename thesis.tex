% !TeX spellcheck = en_GB
\documentclass[abstract,toc,los,english,11pt,glossaries]{jluthesis}


\newacronym{eas}{EAS}{Extensive Air Showers}
\newacronym{uhecr}{UHECR}{Ultra High Energy Cosmic Radiation}
\newacronym{sipm}{SiPM}{Silicon Photomultiplier}
\newacronym{adc}{ADC}{Analog Digital Converter}
\newacronym{enu}{ENU}{East, North Up}
\newacronym{com}{COM}{center of mass}
\newacronym{cdu}{CDU}{cosmic detector unit}
\newacronym{dop}{DOP}{dilution of precision}
\newacronym{csda}{CSDA}{continuous-slowing-down approximation}
\newacronym{gzkl}{GZK-Limit}{Greisen–Zatsepin–Kuzmin limit}
\newacronym{mqtt}{MQTT}{Message Queuing Telemetry Transport}
\newacronym{gnss}{GNSS}{global navigation satellite system}
\newglossaryentry{hadron} {name=hadron, description={subatomic particle affected by the strong interaction}}
\newglossaryentry{nucleus} {name=nucleus, plural=nuclei, description={Center of an atom}}
\bibliography{data/thesis.bib}

\topic{Real-time data analysis of distributed muon detector network}
\title[Bachelor Thesis]{Bachelor Thesis}
\author[Daniel Treffenstädt]{Daniel J.S. Treffenstädt}
\matnr{6067797}
\supervisor{Prof. Dr. Kai-Thomas Brinkmann\\Dr. Hans-Georg Zaunick}




\begin{document}
\makebeginning{
	This thesis attempts to create a real-time analysis of coincidences from timestamped detection events through a network of scintillator detectors. The goal is to create an efficient way to calculate coincidences between all detector stations and identify cosmic air shower events. Ultimately this could lead to the reconstruction of primary cosmic particles which caused the events.
}

\section{Introduction}
\subsection{Cosmic Radiation}
\acrfull{uhecr} is cosmic radiation with energy in the range of $10^{18}$\,eV to $10^{20}$\,eV.\footnote{The energy unit eV(Electron-Volt) describes the kinetic energy a unit charge holds after being accelerated through an electric field for 1V. It is equivalent to $1.602176634\cdot10^{-19}\,\text{J}$.}
Cosmic radiation has been discovered in 1912 by Viktor Hess by performing measurements with three electrometers in a balloon flight. He measured the energy flux in the atmosphere and showed that it increased with altitude. This is explained by secondary particles which are produced through the interaction of primary cosmic radiation with the atmosphere. The cosmic radiation is produced by many different sources, one of which is our sun. The sun however only produces cosmic radiation up to a few hundred MeV of energy through coronal mass ejections and other processes and so is not contributing to \acrshort{uhecr}. Higher energy sources include other stars, supernovae and also black holes. \\
\begin{figure}[ht!]
	\centering
	\includegraphics[width=0.5\linewidth]{data/uhecr-sm-bh}
	\caption{Trajectory of neutral particle around supermassive black hole \cite{Tursunov_2020}}
	\label{fig:uhecr-sm-bh}
\end{figure}\\
The production mechanisms for \acrshort{uhecr} are unknown and an area of active research, though some sources have been speculated. One such proposed mechanism is a spinning supermassive hole with a neutral particle in its accretion disk. The Orbit of this particle decays until it is close enough to the black hole so that the effective electric charge dominates and ionises the particle which causes a large acceleration\cite{Tursunov_2020}. The trajectory of the particle is shown in figure \ref{fig:uhecr-sm-bh}.\\
\begin{figure}[ht!]
	\centering
	\includegraphics[width=0.4\linewidth]{data/trajectory-binary-bh}
	\caption{Possible trajectory of UHECR in Binary Black hole system \cite{Zhang2020}}
	\label{fig:trajectory-binary-bh}
\end{figure} \\
Already high energy particles can be accelerated to the ultra high energy regime. One possible acceleration mechanism relies on a pair of binary black holes which move at relativistic speeds. In such a scenario the particle can experience a series of gravitational slingshots \cite{Zhang2020} around the binary black hole and finally be accelerated so much it reaches the necessary energy region to be classified as a \acrshort{uhecr}. One possible trajectory of a particle in a binary black hole is shown in figure \ref{fig:trajectory-binary-bh}. \\
\begin{figure}[ht!]
	\centering
	\includegraphics[width=0.75\linewidth]{data/cr_spectrum}
	\caption{Spectrum of cosmic radiation \cite{evoli_carmelo_2020_4396125}}
	\label{fig:cr_spectrum}
\end{figure} \\
The spectrum of this cosmic radiation is shown in figure \ref{fig:cr_spectrum}. There, measurements from different experiments are shown. The y axis shows the energy flux of cosmic rays in units of energy per second per square meter per solid angle $\frac{\text{GeV}}{\text{m}^2\cdot\text{s}\cdot\text{sr}}$. The x axis shows the primary energy of the cosmic rays, for comparison the upper axis shows the energy in units of Joule, the lower axis shows it in units of electron-Volt. Notable here is the much higher number of protons as compared to all other particles by two orders of magnitude. This allows one to make the reasonable assumption that most cosmic rays are protons. Prominent features of the proton spectrum are two anomalies labeled as Knee and Ankle. \\
One interesting observation is that the proton spectrum does not have a solid cutoff at $5\cdot10^{19}$\,eV of energy, rather it extends beyond it. This value is significant because it is calculated to be the \acrfull{gzkl}\cite{2021APh...12602526B}. This limit stems from high energy protons interacting with the microwave background radiation and being lifted into an excited state.
\begin{equation}\label{eq:gzkl}
	\begin{aligned}
		\gamma + p &\rightarrow \Delta^+ &\rightarrow p + \pi^0 \\
		\gamma + p &\rightarrow \Delta^+ &\rightarrow n + \pi^+
	\end{aligned}
\end{equation}
Equation \ref{eq:gzkl} descibes the possible interaction between the microwave background photons and protons. In both cases, the proton gets excited to a $\Delta^+$ resonance state, which can decay to a hadron and a pi meson. One possible decay product is a proton and a $\pi^0$, the other one a neutron and a $\pi^+$.
The \acrshort{gzkl} got its name due to its first computation by three physicists independently of each other in 1966. The violation of this limit could be explained with a proposition by the Pierre Auger Observatory that most \acrshort{uhecr} are in fact heavier nuclei than single protons \cite{thepierreaugercollaboration2017inferences}.

\clearpage
\subsection{Cosmic Air Showers}
In the case that a \acrshort{uhecr} interacts with air molecules in the atmosphere, a cascade of secondary particles is created. This shower can be grouped in three components. They are shown in figure \ref{fig:shower-components} and are:
\begin{itemize}
	\item Electromagnetic, photons $\gamma$ and electrons or positrons $e^\pm$
	\item Mesonic, mostly muons $\mu^\pm$
	\item Hadronic, protons $p,\bar{p}$ and neutrons $n,\bar{n}$
\end{itemize}

\begin{figure}[ht!]
	\centering
	\begin{tikzpicture}
		\draw[->] (0,0.5) node[above] {primary particle} -- ++(0,-1.5) coordinate(primary);
		\draw[->] (primary) -- ++(-175:1.5) node[above] {$\pi^0$} -- ++(-175:2) coordinate (pi0);
		\draw[->] (primary) -- ++(-105:1) node[left] {$\pi^\pm$} -- ++(-105:1.5) coordinate (pipm);
		\draw[->] (primary) -- ++(-35:1) -- ++ (-35:2) coordinate (zn);
		
		\draw[->] (pi0) -- ++(-160:1) coordinate(gamma) node[above right] {$\gamma$};
		\draw[->] (pi0) -- ++(-130:1) node[below right] {$\gamma$};
		\draw[->] (gamma) -- ++(-165:1) node[above] {$e^\pm$};
		\draw[->] (gamma) -- ++(-155:1) node[below] {$e^\mp$};
		
		\draw[->] (pipm) -- ++(-110:1) node[left] {$\mu^\pm$};
		
		\draw[->] (zn) -- ++(-10:1) node[above] {p};
		\draw[->] (zn) -- ++(-30:1) node[below] {n};
		
		\draw[thin, dashed] (-2,0) -- ++(0,-5) ++ (-0.5,0) node[left] {Electromagnetic};
		\draw[thin, dashed] (2,0) -- ++(0,-5) ++(-1,0) node[left] {Mesonic} ++(1.5,0) node[right] {Hadronic};
	\end{tikzpicture}
	\caption{Components of a cosmic air shower}
	\label{fig:shower-components}
\end{figure}
A class of cosmic air showers are the \acrfull{eas}, those are showers produced by \acrshort{uhecr}. A picture of a simulated air shower produced by a primary proton is shown in figure \ref{fig:proton-shower}. This image is produced at the KIT with the simulation toolkit CORSIKA. 
\begin{figure}[ht!]
	\centering
	\includegraphics[width=0.8\linewidth]{data/shower-45}
	\caption{Proton Air Shower \cite{corsika-images}}
	\label{fig:proton-shower}
\end{figure}
\todo{Decay channels}
\clearpage
\subsection{Detection methods}
\todo{scintillators}
\acrshort{eas} can be detected through a number of methods. One such method is to observe the Cherenkov radiation produced by the primary particle. This radiation occurs since the local speed of light in the visible spectrum is lower than its vacuum speed. Due to the high energy of the primary particle its speed is nearly $1c_0$. This causes an effect comparable to a mach cone for supersonic objects in air and is known as Cherenkow radiation. This radiation is in the UV spectrum and can be observed at the surface of earth for example with an array of sensitive telescopes. One such array is HESS, located in Namibia. \\
Also detectable is the electromagnetic component of the showers, since its spectrum has a peak in the radio range and this penetrates deeply into the atmosphere. This can be picked up by a range of radio antenna. \\
Another method is to detect the muonic component. Muons ($\mu$) are elementary particles, often called the heavy sibling of the electron. It also has a charge of $-1$, a spin of $\frac{1}{2}$ but a much higher mass of
$105.6583755\,\si{MeV/c^2}$. It also only has a mean lifetime of $\snsi{2.1969811}{-6}{s}$. Muons only interact weakly with matter and thus penetrate deeply into the atmosphere. This fact has been used in the past to image large masses such as volcanoes and pyramids by a method called muon tomography.
\subsection{Range of Muons in Air}
\section{Detector Network}
The detector network is made up of a number of low-cost scintillator detectors. Each detector consists of a scintillator with \acrfull{sipm} assembly, an interface board with \acrfull{adc} and a RaspberryPi minicomputer. The interface board analyses signals from the \acrshort{sipm}, retrieves the current timestamp and passes this timestamped event along to a software running on the RaspberryPi minicomputer which sends it to a server. Each timestamped event is thus written to a central database where it can be further analysed at a later time. Due to the number of detectors and their respective rate of few Hz, a large number of events is writen to the database. This causes a large strain on the server infrastructure when running post-analysis programs since they need to handle large amounts of data in a short period of time. Since this is sub-optimal, a realtime solution which performs as much analysis and pre-filtering as possible in order to reduce the runtime impact of the analysis is preferable.

\section{Real-time analysis}
\subsection{Coincidence criteria}
A sturdy criterium to define a coincidence has to be found. Initially the criterium was defined as the $\Delta{t}$ of two events being lower than $100\,\mu\text{s}$, which corresponds to a maximum possible detector distance of $29.97\,\text{km}$. Since in reality the maximum coincidence interval is dependent on the actual detector distance and is limited by the flight distance of a muon, this criterium must be improved upon. A first approach lies in calculating the distance between the detector pair belonging to the coincidence event which is to be analysed. This is implemented as the euclidean distance derived from the detector coordinates. Therefore the WGS84 earth coordinate model has been implemented to be able to convert a given pair of geodetic coordinates to relative carthesian coordinates, specifically in the \acrfull{enu} system. The WGS84 model thereby approximates the shape of earth as an oblate spheroid. Given the straight line distance, the time of flight given the speed of light in vacuum $c_0$ can be taken as a first approximation for the expected maximum coincidence window.

This value however still needs to be refined in two ways. firstly, a minimum value has to be set due to limitations of the possible time resolution of the timestamping of each event and a not well understood per-detector timing offset on the order of magnitude of few tens of ns. This is especially critical in the case of detectors which are close together, since their calculated time of flight is only few ns, and thus would cut off a significant number of coincidence events. In order to mitigate this, a minimum coincidence window of $150\,\text{ns}$ width is defined. This value is chosen to reliably capture all coincidence events for even the closest detector pairs.

A maximum upper coincidence window width has to be defined too, though this requires a more in depth analysis. Following influences have to be considered:
\begin{itemize}
	\item maximum expected zenith angle of muons
	\item maximum flight distance for a muon of maximum expected energy
	\item Muon flux cutoff value
\end{itemize}
\subsubsection*{Zenith angle}
The muon flux after the zenith angle can be described by \todo{function for distribution} distribution\cite{muonenergyspectrum} which can be approximated by a $\cos^2$ function. A cutoff of relative intensity of $1\,\%$ is chosen for the maximum zenith angle which leads to an angle of $\theta_{max} = \arccos\left(\sqrt{1\,\%}\right) = 84\,^\circ$.
\subsubsection*{Energy cutoff}
\subsubsection*{Muon Flux cutoff}
The muon flux cutoff value is chosen so that the frequency of the muons which are lost is below a threshold of $1\,\frac{1}{\text{a}}$. This frequency is determined through the muon spectrum\cite{muonenergyspectrum} at a zenith angle of $\theta = 0^{\circ}$. 
\subsection{Coincidence and plausability determination}
In order to determine coincidences, each event needs to be checked against all other events which are currently in the event buffer. To do this, the
\subsection{Conflicting data resolution}
\clearpage 
\section{Program layout}
The path of an event in the program is shown in figure \ref{fig:programlayout}. The source receives the event data and passes it on to the Station supervision class. It supervises the runtime data of each detector station and classifies it into reliability states. The specific criteria are described in the flowchart in figure \todo{make flowchart}.
\begin{figure}[ht!]
	\centering
	\begin{tikzpicture}[->]
		\node (source) [draw, process,minimum height=1cm] {Event Source};
		\node (supervision) [below=of source, draw, process,minimum height=1cm] {Station supervision};
		\node (filter) [below=of supervision, draw, process,minimum height=1cm] {Coincidence filter};
		\node (refinement) [below=of filter, draw, process,minimum height=1cm] {Coincidence refinement};
		\node (sink) [below=of refinement, draw, process,minimum height=1cm] {Event Sink};

		\node (criterium) [left=of filter, draw, process,minimum height=1cm] {Coincidence criterium};
		
		\draw (source) -- (supervision);
		\draw (supervision) -- (filter);
		\draw (filter) -- (refinement);
		\draw (refinement) -- (sink);
		
		\draw[<->] (criterium) -- (filter);
	\end{tikzpicture}
	\caption{Program layout}
	\label{fig:programlayout}
\end{figure}

\begin{figure}[ht!]
	\centering
	\begin{tikzpicture}[->]
		\node (terminal1) at (0,0) [draw, terminal] {START};
		\node (decide1) [below=of terminal1, draw, decision] {detector known};
		\node (predproc1) [below=of decide1, draw, predproc, align=left] {calculate event rates};
		\node (decide2) [below=of predproc1, draw, decision] {detector is reliable};
		\node (terminal2) [below=of decide2, draw, terminal] {pass to coincidence filter};
		\node (terminal3) [right=of predproc1, draw, terminal] {discard event};
		
		\node (decide1r) [right=of decide1] {no};
		\node (decide2r) [right=of decide2] {no};
		
		
		\draw (terminal1) -- (decide1);
		\draw (decide1) -- (predproc1);
		\draw (predproc1) -- (decide2);
		\draw (decide2) -- (terminal2);
		\draw (decide1) -- (decide1r) -| (terminal3);
		\draw (decide2) -- (decide2r) -| (terminal3);
	\end{tikzpicture}
\end{figure}
\begin{figure}[ht!]
	\centering
	\begin{tikzpicture}[->]
		\node (terminal1) at (0,0) [draw, terminal] {START};
		\node (decide1) [below=of terminal1, draw, decision] {constructor available};
		\node (predproc1) [below=of decide1, draw, predproc, align=left] {select next constructor};
		\node (decide2) [below=of predproc1, draw, decision] {criterium met};
		\node (predproc2) [below=of decide2, draw, predproc] {add event to constructor};
		\node (terminal2) [below=of predproc2, draw, terminal] {END};
		
		\node (decide1r) [right=of decide1] {no};
		\node (decide2l) [left=of decide2] {no};
		
		
		\draw (terminal1) -- (decide1);
		\draw (decide1) -- (predproc1);
		\draw (predproc1) -- (decide2);
		\draw (decide2) -- (predproc2);
		\draw (predproc2) -- (terminal2);
		\draw (decide1) -- (decide1r) |- (terminal2);
		\draw (decide2) -- (decide2l) |- (decide1);
	\end{tikzpicture}
\end{figure}
\begin{figure}[ht!]
	\centering
	\begin{tikzpicture}[->]
		\node (terminal1) at (0,0) [draw, terminal] {START};
		\node (decide1) [below=of terminal1, draw, decision] {constructor is timed out};
		\node (predproc1) [below=of decide1, draw, predproc, align=left] {select next constructor};
		\node (decide2) [below=of predproc1, draw, decision] {criterium met};
		\node (predproc2) [below=of decide2, draw, predproc] {add event to constructor};
		\node (terminal2) [below=of predproc2, draw, terminal] {END};
		
		\node (decide1r) [right=of decide1] {no};
		\node (decide2l) [left=of decide2] {no};
		
		
		\draw (terminal1) -- (decide1);
		\draw (decide1) -- (predproc1);
		\draw (predproc1) -- (decide2);
		\draw (decide2) -- (predproc2);
		\draw (predproc2) -- (terminal2);
		\draw (decide1) -- (decide1r) |- (terminal2);
		\draw (decide2) -- (decide2l) |- (decide1);
	\end{tikzpicture}
\end{figure}
\section{Primary reconstruction}
\subsection{Possible primary incident angles}
Through the assumptions shown in figure \ref{fig:angle-assumptions}, the possible source directions of a primary particle can be reduced. There, the geometry of a coincidence with $n=2$ is shown. Since she shower front can be simplified as a spherical section and the propagation speed can be assumed to be near the speed of light, two spheres with a shared tangent can be constructed. One spherical surface is the shower front, the other is a sphere around the detector which had the later event timing. Its radius $\ell$ is equal to the propagation time of light for the coincidence time. Through geometrical analysis of this arrangement, the Radius and angle pair can be derived as a function of the angle $\alpha$. This creates a curve in the plane on which the primary interaction must have taken place. Due to the symmetry of this arrangement however, in 3-D space, this curve becomes a rotational surface. \\
In the case of a coincidence with $n>2$ multiple curves can be calculated relative to the same root detector. This allows for the intersection points of each rotational surface to be calculated, which results in a reduced possible origin area for the primary interaction. \\
Given that the shower front is not a complete spherical surface but only a spherical section, the possible origin directions can be reduced with this approach. In the case of higher coincidence numbers, the possible direction can be reduced further. Additionally to the possible directions, this approach also provides a lower limit for the shower radius and footprint radius, suggesting a minimum Particle energy.
\begin{figure}[ht!]
	\centering
	\begin{tikzpicture}
		\fill (0,0) coordinate(det1) circle(0.1);
		\fill (6,0) coordinate(det2) circle(0.1);
		\draw (det2) circle(1);
		\draw (det2) -- ++(125:0.7) node[right] {$\ell$} -- ++(125:6.4) coordinate(center);
		\draw[dashed] (det2) ++(125:1) ++ (215:2) -- ++(35:4) node[right] {tangent};
		\draw (det2) ++ (125:1) arc(-55:-35:6.1) arc(-35:-125:6.1) node[left] {shower front};
		\draw (center) -- (det1) -- (det2);
		\draw (center) -- ++(0,-6.1);
		\draw[<->] (center) ++(-55:1) arc(-55:-90:1);
		\draw[<->] (center) ++(-55:1) arc(-55:-90:1);
		\draw (det2) ++(125:1) -- (det1);
		\draw (det2) ++(125:1) -- ++(0,1) -- ++(0,-2);
		
		\draw[<->] (det2) ++(125:0.25) arc(-55:-90:0.75) node[above right] {$\alpha$};
		
	\end{tikzpicture}
	\caption{Assumption for possible incidence angles}
	\label{fig:angle-assumptions}
\end{figure}
\begin{equation}
	\gamma = \frac{\pi}{2} - \alpha - \arctan\left(\frac{\ell\cdot\cos\alpha}{d - \ell\sin\alpha}\right)
\end{equation}
\begin{equation}
	\frac{\delta}{2} = \frac{\pi}{2} - \gamma = \alpha + \arctan\left(\frac{\ell\cdot\cos\alpha}{d - \ell\sin\alpha}\right)
\end{equation}
\begin{equation}
	\begin{aligned}
	\xi &= \frac{\pi}{2} - \delta - \left(\gamma + \beta\right) = \frac{\pi}{2} - 2\left(\alpha + \arctan\left(\frac{\ell\cdot\cos\alpha}{d - \ell\sin\alpha}\right)\right) - \left(\frac{\pi}{2} - \alpha\right) \\
	&= -3\alpha - 2\arctan\left(\frac{\ell\cdot\cos\alpha}{d - \ell\sin\alpha}\right)
\end{aligned}
\end{equation}
\begin{equation}
	r = \frac{s}{2\cdot\sin\left(\frac{\delta}{2}\right)}
\end{equation}
\clearpage
\section{Conclusion}
\clearpage
\backmatter
\end{document}
